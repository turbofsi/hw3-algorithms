\section{Problem I}
\textbf{solution}:\\
\begin{enumerate}
	\item \textbf{proof:} \\
	Without loss of generality, suppose that $I_m > I_t$ are two elements in $\langle I_1, I_2, ..., I_k \rangle$ where $1 \leq m \geq k$, $ 1 \leq t \leq k $ and $m \neq t$. Suppose that Greedy-Schedule determines order $I_m \succ I_t$ on $I$. Greedy-Schedule determines an order $I_m \prec I_t$ on $I'$.\\

	From definition of order restriction, for $I_m, I_t \in I$ and $I_m, I_t \in I'$. $I_m$ precedes $I_t$ \textbf{iff} it also determines an order on $I'$ where $I_m$ precedes $I_t$ yielding a contradiction to the assumption that $I_m \succ I_t$ on $I$, $I_m \prec I_t$ on $I'$. The opposite direction is the same. Thus, Greedy-Schedule determines an order $I_1 \succ I_2 \succ ... \succ I_k$ on $I$ if Greedy-Schedule determines an order $I_1 \succ I_2 \succ ... \succ I_k$ on $I'$. 
	\item \textbf{proof:}\\
	we first prove if (a), (b) and (c) happens $\Rightarrow$ Greedy-Schedule is not correct.\\

	If we use algorithm First Starting Time First which means we choose interval in $I$ with the earliest start time. Let interval $I[a], I[b], I[c] \in I $ where $s_a < s_b < s_c$ and $f_c < f_a$ and $f_b < s_c$. For these three intervals, they meet (a), (b), (c) as shown in the figure.\\

	If we choose interval in $I$ with the earliest start time, then we will choose $I_a$ instead of $I_b$ and $I_c$. For optimal schedule, we should choose $I_b$ and $I_c$ instead of $I_a$ since we could schedule more intervals within the same time slot. Thus, Greedy-Schedule is not optimal. \\

	Greedy-Schedule is not correct $\Rightarrow$ (a), (b), (c) happens. \\

	If Greedy-Schedule is not optimal, which means it will choose less intervals compare with the optimal set. If Greedy-Schedule $GS$ is not optimal and $I_a, I_b, I_c \in I$. As $GS$ is not optimal we can assume that $GS$ picked $I_a$ instead of $I_b, I_c$. But for the optimal schedule set($OS$), it choose $I_b$ and $I_c$ instead of $I_a$. Which means that $I_a$ should overlap with $I_b$ and $I_c$. So they could not be chosen at the same time. Which meets the situation (a). As $I_b$ and $I_c$ could be chosen together which implies that $I_b$ and $I_c$ does not overlap with each other. So it also meets situation (b). As in $GS$, $I_a$ is chosen first, so we could not choose $I_b$ and $I_c$. Which implies that on input I Greedy-Schedule determines an ordering $O$ where $I[a]$ comes before $I[b]$ and $I[c]$, so situation(c) also happens. 

	\item \textbf{proof:} \\
	According to the conclusion in question (2), we know that Greedy-Schedule is correct iff (a), (b), (c) not happens. So to prove LSTF is correct, is equivalent to prove (a), (b) and (c) cannot happen. \\

	Suppose that for LSTF, (a), (b), (c) happens. For $I_a, I_b, I_c \in I$. We let $s_i, f_i$ be the start time and finish time of interval $I_i$. If (a) is true, here without loss of generality, we let $I_b$ precedes $I_c$,  then $s_a < f_b$ (1) and $ s_c < f_a $. If (b) is true. $f_b < s_c$ (2). From (1) and (2) we know that $s_a < f_b < s_c$ (3). If (c) is true, according to the rules of LSTF, as $I_a$ precedes $I_b, I_c$, $s_a > s_b > s_c$ which contradict (3). So, for LSTF (a), (b), (c) cannot happen together. Thus LSTF is correct. 
\end{enumerate}